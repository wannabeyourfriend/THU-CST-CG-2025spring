\documentclass[a4paper,twoside]{article}
\usepackage[UTF8, scheme = plain]{ctex}%提供中文支持的包,与XeLaTeX一起使用
\usepackage{algorithm}
\usepackage{algorithmic}
\usepackage{minted}% 语法高亮和代码样式设置方面更加强大和灵活
\usepackage{listings}% 引入listings包,用于在文档中插入代码,并可自定义代码样式
\usepackage[a4paper, total={7in, 11in}, top=0.1in]{geometry}

\title{Computer Graphic PA0}
\author{
  王子轩\\
  \texttt{2023011307} \\
  \texttt{wang-zx23@mails.tsinghua.edu.cn} \\
}
\date{2025/3/15}

\begin{document}
\maketitle
\section{Algorithm}
\subsection{线段绘制}
首先,利用最原始的Bresenham画线算法,可以画出一条从起点$(x_0,y_0)$到终点$(x_1, y_1)$的线段,其中起点需要在终点的左下方.但该过程需要利用除法计算$k$,实际上不是必须的.
\begin{algorithm}[H]
    \caption{Basic Bresenham Line Drawing Algorithm}
    \label{alg:bresenham}
    \begin{algorithmic}[1]
        \REQUIRE Starting point $(x_0, y_0)$, Ending point $(x_1, y_1)$, Image object, and color
        \ENSURE Draw a line from $(x_0, y_0)$ to $(x_1, y_1)$ in the specified color

        \STATE Calculate the differences: $dx = x_1 - x_0$, $dy = y_1 - y_0$
        \STATE Calculate the slope: $k = \frac{dy}{dx}$
        \STATE Initialize the error term: $e = -0.5$
        \STATE Set the starting position: $x = x_0$, $y = y_0$

        \FOR{each $i$ from 0 to $dx$}
            \STATE Set the pixel at $(x, y)$ to the given color in the image
            \STATE Increment $x$ by 1
            \STATE Update the error term: $e = e + k$
            \IF{$e \geq 0$}
                \STATE Increment $y$ by 1
                \STATE Decrease the error term by 1: $e = e - 1$
            \ENDIF
        \ENDFOR
    \end{algorithmic}
\end{algorithm}
我们将该算法进行改进:通过计算起点和终点的坐标差值(dx, dy),确定绘制过程中每一步在x和y方向上的变化量.根据起点和终点的相对位置,代码决定了每次x和y坐标的增减步长(sx和sy).然后,设置一个初始误差值(err),并进入循环,根据误差值调整x和y的坐标.每次迭代中,当前像素点(x0, y0)会被绘制,如果当前点已经到达终点,则退出循环.如果没有到达终点,算法根据误差值判断是更新x坐标还是y坐标.
\begin{listing}[H]
	\caption{实现 void Line::draw()}
	\label{code:processdweet}
	\begin{minted}{cpp}
    void draw(Image &img) override {
        int x0 = xA, y0 = yA, x1 = xB, y1 = yB;
        int dx = abs(x1 - x0), dy = -abs(y1 - y0);
        int sx = (x0 < x1) ? 1 : -1, sy = (y0 < y1) ? 1 : -1;
        int err = dx + dy;
        while (true){
            img.SetPixel(x0, y0, color);
            if (x0 == x1 && y0 == y1) break;
            int e2 = 2 * err;
            if (e2 >= dy){ 
                if (x0 == x1) break;
                err += dy; x0 += sx; 
            }
            if (e2 <= dx) {
                if (y0 == y1) break;
                err += dx; y0 += sy;  
            }
        }
    }
\end{minted}
\end{listing}
\subsection{圆弧绘制}
使用中点画圆算法(Midpoint Circle Algorithm)来绘制圆.该算法基于圆的八分对称性,只需要计算圆的八分之一的点,然后通过对称变换得到整个圆的所有点.
\begin{algorithm}[H]
    \caption{Midpoint Circle Drawing Algorithm}
    \label{alg:midpoint_circle}
    \begin{algorithmic}[1]
        \REQUIRE Circle center $(cx, cy)$, radius $r$, Image object, and color
        \ENSURE Draw a circle with center $(cx, cy)$ and radius $r$ in the specified color

        \STATE Initialize: $x = 0$, $y = r$
        \STATE Calculate initial decision parameter: $d = 1.25 - r$
        \STATE Draw points at $(cx \pm x, cy \pm y)$ and $(cx \pm y, cy \pm x)$
        
        \WHILE{$x \leq y$}
            \IF{$d < 0$}
                \STATE Update decision parameter: $d = d + 2x + 3$
            \ELSE
                \STATE Update decision parameter: $d = d + 2(x - y) + 5$
                \STATE Decrement $y$ by 1
            \ENDIF
            \STATE Increment $x$ by 1
            \STATE Draw points at $(cx \pm x, cy \pm y)$ and $(cx \pm y, cy \pm x)$
        \ENDWHILE
    \end{algorithmic}
\end{algorithm}
算法使用一个决策参数$d$来决定下一个像素点的位置,避免了浮点数运算.当$d < 0$时,选择水平方向的像素点;当$d \geq 0$时,选择对角线方向的像素点.决策参数d的推导过程:考虑圆的方程:$F(x,y) = x^2 + y^2 - r^2 = 0$,在点$(x_k, y_k)$处,我们需要决定下一个点是选择$(x_k+1, y_k)$还是$(x_k+1, y_k-1)$.中点判别法考虑这两个候选点的中点$(x_k+1, y_k-\frac{1}{2})$.定义决策参数:$d_k = F(x_k+1, y_k-\frac{1}{2})$,$= (x_k+1)^2 + (y_k-\frac{1}{2})^2 - r^2$当$d_k < 0$时,中点在圆内,选择$(x_k+1, y_k)$;
当$d_k \geq 0$时,中点在圆外或圆上,选择$(x_k+1, y_k-1)$.下一个决策参数$d_{k+1}$的递推关系:当$d_k < 0$时(选择E点):
   $d_{k+1} = F(x_k+2, y_k-\frac{1}{2})$
   $= d_k + 2(x_k+1) + 1$
   $= d_k + 2x_k + 3$
2. 当$d_k \geq 0$时(选择SE点):
   $d_{k+1} = F(x_k+2, (y_k-1)-\frac{1}{2})$
   $= d_k + 2(x_k+1) - 2(y_k-1) + 1$
   $= d_k + 2(x_k - y_k) + 5$
这就是代码中更新决策参数d的依据.初始值$d_0 = 1.25 - r$是将$k=0$代入决策参数公式得到的.
\begin{listing}[H]
    \caption{实现 void Circle::draw()}
    \label{code:circle_draw}
    \begin{minted}{cpp}
    void draw(Image &img) override {
        int x = 0, y = radius; float d = 1.25 - radius;
        auto drawCirclePoints = [&](int x, int y) {
            img.SetPixel(cx + x, cy + y, color); img.SetPixel(cx - x, cy + y, color);
            img.SetPixel(cx + x, cy - y, color); img.SetPixel(cx - x, cy - y, color);
            img.SetPixel(cx + y, cy + x, color); img.SetPixel(cx - y, cy + x, color);
            img.SetPixel(cx + y, cy - x, color); img.SetPixel(cx - y, cy - x, color);
        };
        drawCirclePoints(x, y);
        while (x <= y) {
            if (d < 0)  d += 2 * x + 3
            else { 
                d += 2 * (x - y) + 5;
                y--;
            }
            x++;
            drawCirclePoints(x, y);
        }
    }
    \end{minted}
\end{listing}
\subsection{颜色填充}
洪水填充算法(Flood Fill Algorithm)是一种用于填充连通区域的算法.该算法从一个起始点开始,通过广度优先搜索的方式,将所有与起始点颜色相同且相连的像素点都替换为目标颜色.

\begin{algorithm}[H]
    \caption{Flood Fill Algorithm}
    \label{alg:flood_fill}
    \begin{algorithmic}[1]
        \REQUIRE Starting point $(cx, cy)$, target color $color$, Image object $img$
        \ENSURE Fill the connected region starting from the given point

        \STATE Get the color at starting point: $targetColor = img(cx, cy)$
        \STATE If $targetColor = color$ then return
        \STATE Create queue $Q$ and add starting point $(cx, cy)$ to $Q$
        
        \WHILE{$Q$ is not empty}
            \STATE Get front pixel $(x, y)$ from $Q$
            \IF{$(x, y)$ is out of bounds OR color at $(x, y)$ $\neq targetColor$}
                \STATE Continue to next iteration
            \ENDIF
            \STATE Set color at $(x, y)$ to $color$
            \STATE Add $(x+1, y)$, $(x-1, y)$, $(x, y+1)$, $(x, y-1)$ to $Q$
        \ENDWHILE
    \end{algorithmic}
\end{algorithm}
采用广度优先搜索策略,使用队列存储待处理的像素点.对于每个待处理的像素点,首先检查其是否越界或是否需要填充(颜色是否与目标颜色相同).如果需要填充,则将其颜色更改为目标颜色,并将其四个相邻像素点(上,下,左,右)加入队列.这个过程会一直持续到队列为空,即所有需要填充的像素都已处理完毕.
\begin{listing}[H]
    \caption{实现 void Fill::draw()}
    \label{code:fill_draw}
    \begin{minted}{cpp}
    void draw(Image &img) override {
        Vector3f targetColor = img.GetPixel(cx, cy);
        if (targetColor == color) {
            return;
        }
        std::queue<std::pair<int, int>> pixels;
        pixels.push({cx, cy});
        int width = img.Width();
        int height = img.Height();
        while (!pixels.empty()) {
            auto [x, y] = pixels.front();
            pixels.pop();
            if (x < 0 || x >= width || y < 0 || y >= height ||
                img.GetPixel(x, y) != targetColor) {
                continue;
            }
            img.SetPixel(x, y, color);
            pixels.push({x + 1, y}); pixels.push({x - 1, y}); 
            pixels.push({x, y + 1}); pixels.push({x, y - 1}); 
        }
    }
    \end{minted}
\end{listing}

\section{Honor Code}
PA0算法思路部分借鉴了<计算机图形学基础>(第二版)P22-P31内容,就代码实现时的bug与Claud-Sonnet 3.5模型进行了交互,并借鉴了其部分做法
\section{Problem and Solution}
\subsection{ $k= \frac{dy}{dx}$当$dx=0$ 以及起止点相对位置讨论}
在1.1部分的代码实现中,我本来是利用的原版的Bresenham画线算法,需要计算直线的斜率,然而没有考虑到除法分母为零的竖直直线情形,从而产生了错误的代码逻辑,产生了如图所示的测试例子;同时对起止点的位置进行讨论因此绘图没有正确处理.解决办法:采用改进后的算法,改用整数避免除法,做$e_2 = 2 * err$的替换,并引入sx和sy控制起止点相对位置与绘制点自增方向的关系.
\section{ Suggestion}
一个是建议可以给更多的测试例子;第二个是好像这三个算法都比较简单(?),也许可以增加一些反走样的实验.感觉上小作业难度和大作业难度跨越较大(?)上课内容感觉相对前沿,但上课内容和作业似乎线性无关.
\end{document}
